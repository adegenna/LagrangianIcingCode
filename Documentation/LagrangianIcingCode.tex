\documentclass{article}
\usepackage[margin=1in]{geometry}
\usepackage{fancyhdr}
\usepackage{indentfirst}
\usepackage{graphicx}
\usepackage{subfigure}
\usepackage{subfigmat}
\usepackage{bm, amssymb, amsmath, array, pdfpages}
\usepackage{listings}

\usepackage{titlesec}

\graphicspath{{Figures/}}

% define some commands to maintain consistency
 \newcommand{\pkg}[1]{\texttt{#1}}
 \newcommand{\cls}[1]{\textsf{#1}}
 \newcommand{\file}[1]{\texttt{#1}}
\newcommand{\todo}[1]{\textcolor{red}{[TODO: #1]}}
\newcommand{\adgremark}[1]{\textcolor{blue}{[ADG: #1]}}
\newcommand{\R}{\mathbb{R}}
\newcommand{\CLmax}{{C_L}_\text{max}}
\newcommand{\LDmax}{L/D_\text{max}}
\newcommand{\alphamax}{\alpha_\text{max}}
\def\ip<#1,#2>{\left\langle #1,#2\right\rangle}

%%%%%%%%%%%%%%%%%%%%%%%%%%%%%%%%%%%%%%%%%%%%%%%%%%%%%%%%%%
% MODIFICATIONS TO ALLOW HEADER ON THE TITLE PAGE
%%%%%%%%%%%%%%%%%%%%%%%%%%%%%%%%%%%%%%%%%%%%%%%%%%%%%%%%%%

\makeatletter
% Copied from article.cls
\renewcommand\maketitle{\par
  \begingroup
    \renewcommand\thefootnote{\@fnsymbol\c@footnote}%
    \def\@makefnmark{\rlap{\@textsuperscript{\normalfont\@thefnmark}}}%
    \long\def\@makefntext##1{\parindent 1em\noindent
            \hb@xt@1.8em{%
                \hss\@textsuperscript{\normalfont\@thefnmark}}##1}%
    \if@twocolumn
      \ifnum \col@number=\@ne
        \@maketitle
      \else
        \twocolumn[\@maketitle]%
      \fi
    \else
      \newpage
      \global\@topnum\z@   % Prevents figures from going at top of page.
      \@maketitle
    \fi
    \thispagestyle{fancy}\@thanks % was {empty}
  \endgroup
  \setcounter{footnote}{0}%
  \global\let\thanks\relax
  \global\let\maketitle\relax
  \global\let\@maketitle\relax
  \global\let\@thanks\@empty
  \global\let\@author\@empty
  %\global\let\@date\@empty
  \global\let\@title\@empty
  \global\let\title\relax
  \global\let\author\relax
  %\global\let\date\relax
  \global\let\and\relax
}
% end of copy
\makeatother

\newcommand*{\vertbar}{\rule[1ex]{0.5pt}{2.5ex}}
\newcommand*{\horzbar}{\rule[.5ex]{2.5ex}{0.5pt}}

%%%%%%%%%%%%%%%%%%%%%%%%%%%%%%%%%%%%%%%%%%%%%%%%%%%%%%%%%%
% TITLE, DATE, AUTHOR, HEADERS
%%%%%%%%%%%%%%%%%%%%%%%%%%%%%%%%%%%%%%%%%%%%%%%%%%%%%%%%%%


\begin{document}
\pagestyle{fancy}
\lhead{\textit{\bf Notes: A Lagrangian Spray Model for Airfoil Icing}}
\rhead{\bf February 2015} %empty
\title{\LARGE \bf
Notes: A Lagrangian Spray Model for Airfoil Icing}
\author{Anthony DeGennaro}
\date{July 2014}
\maketitle

%%%%%%%%%%%%%%%%%%%%%%%%%%%%%%%%%%%%%%%%%%%%%%%%%%%%%%%%%%
% ANTHONY DEGENNARO'S PROPOSAL
%%%%%%%%%%%%%%%%%%%%%%%%%%%%%%%%%%%%%%%%%%%%%%%%%%%%%%%%%%

\section{Introduction and Motivation}
The purpose of these notes is to describe the ongoing development of a wing ice accretion code 
which applies techniques from the liquid spray community to improve on existing methods.

It is well-known that individual droplets in the air can significantly affect ice accretion on wings.
As an example, an ATR-72 crashed in Roselawn, Indiana in 1994; it is speculated that this incident 
was largely due to Supercooled-Liquid Droplets (SLDs) in the air. This crash has led to a revamping 
of the FAA regulations on flight conditions to explicitly consider freezing drizzle/rain.

Existing computational codes for icing generally do {\it not} track each individual droplet impinging 
on the wing throughout the course of a simulation. The reason for this is simple: there are too many 
droplets for this to be computationally feasible. We can demonstrate this easily with a back of the 
envelope calculation: assuming the liquid water content (LWC) of the air to be 0.3 g/m$^3$, the 
airspeed to be 100 m/s, and the mean volumetric diameter (MVD) of the liquid droplets to be 50 $\mu$m, 
this gives a free-stream flux of about 500 million particles per square meter per second. Since wing icing 
simulations can frequently last 5-10 minutes or longer, one is faced with the proposition of tracking (potentially)
billions of particles over the course of one simulation (this is not even accounting for new droplets created 
through the process of droplet splashing and breakup).

Existing methods deal with this problem in a few ways. First, many methods do not employ a Lagrangian 
approach; rather, they treat the liquid/air as a two-phase continuum and use Eulerian formulations, which 
automatically circumvents the need to track all particles. Regardless, most methods assume that the 
liquid mass impinging on the wing surface can be modeled as a continuum (the so-called ``collection 
efficiency"). Because of this assumption, there is no need to advect the billions of particles one would 
do in a fully time-resolved Lagrangian simulation; instead, one need only advect an initial ``sheet" of 
particles, observe how that sheet impinges on the wing surface, and calculate a mass flux ratio that way. 
Of course, another notable assumption here is that the collection efficiency does not vary over the 
course of the simulation, even though the droplet properties may be randomly distributed.

Both of these solutions lead to fast simulations at the expense of ignoring (1) the statistical effect of 
having droplet properties which are randomly distributed and (2) the effect on the ice shape of individual 
droplets. This is where we hope to make a contribution. Borrowing from methods used in the spray 
community, we formulate the icing problem probabilistically, by equipping the freestream droplet 
properties with a probability density function. We then formulate a Lagrangian in which the advected 
particles are actually ``clumps" of droplets. This keeps the number of parcels tracked in each simulation 
relatively low. In order to investigate the statistical effects of the droplet distribution, one need only generate 
a multirealization ensemble of simulations. We also hope to develop a thermodynamic model that governs the 
behavior of the individual droplets as they impinge and freeze on the surface; this would potentially help 
towards investigating the effect of individual droplets on ice shape.


\section{Brief Sketch of Method}
Writing the interpolation matrix as $\bf \Phi_{X}$, and the vector of response surface values 
as $\bf Y$, we have the regression problem $\bf \Phi_{X}^T c = Y$. Note that ${\bf \Phi_X} \in \mathbb{R}^{K \times Q}$ 
and ${\bf Y} \in \mathbb{R}^{Q \times 1}$.

A least-squares solution of this system would involve calculating the psuedo-inverse: $\bf c = Y (\Phi_{X}^T)^{\dagger}$. 

First, we show how to reduce the dimensionality of the calculation through a QR factorization. Define:
\begin{equation}
{\bf \Phi_{XY}} = 
\begin{bmatrix}
\bf \Phi_X \\
\bf Y^T
\end{bmatrix}
\in \mathbb{R}^{(K+1) \times Q}
\end{equation}
We use the QR factorization to write $\bf \Phi_{XY}^T = QR = QL^T$ with: 
\begin{equation}
\bf L = 
\begin{bmatrix}
\bf L_{11} & \bf 0 \\
\bf L_{21} & \bf L_{22}
\end{bmatrix}
\end{equation}

This gives us $\bf (\Phi_X)^T = Q [ L_{11} \;\;\; 0 ]^T $ and $\bf Y = Q [ L_{21} \;\;\; L_{22} ]^T  $. Thus:
\begin{equation}
\begin{aligned}
\bf c = Y (\Phi_{X}^T)^{\dagger} &= \left \lbrace \bf Q
\begin{bmatrix}
\bf L_{21}^T \\
\bf L_{22}^T
\end{bmatrix} \right \rbrace
\left \lbrace \bf Q
\begin{bmatrix}
\bf L_{11}^T \\
\bf 0
\end{bmatrix} \right \rbrace ^{\dagger} \\
& = 
\left \lbrace \bf Q
\begin{bmatrix}
\bf L_{21}^T \\
\bf L_{22}^T
\end{bmatrix} \right \rbrace
\left \lbrace
\begin{bmatrix}
\bf L_{11}^{-T} & \bf 0
\end{bmatrix} \bf Q^T
\right \rbrace \\
& = 
\bf Q 
\begin{bmatrix}
\bf L_{21}^T L_{11}^{-T} \;\;\;\;\; 0 \\
\bf L_{22}^T L_{11}^{-T} \;\;\;\;\; 0
\end{bmatrix} \bf Q^T
\end{aligned}
\end{equation}

This matrix has a transpose which is upper block triangular, and so its eigenvalues 
are given as the union of those of the transposed diagonal blocks, which are zeros together with 
the eigenvalues of $\bf (L_{11}^{-1}) L_{21} $.



\end{document}